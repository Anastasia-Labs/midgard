\documentclass[../midgard.tex]{subfiles}
\graphicspath{{\subfix{../images/}}}
\begin{document}

\section{Withdrawal order (L1)}
\label{h:withdrawal-order}

\notebox{Currently, this withdrawal order design can only handle withdrawing pubkey-based utxos.
However, we can modify L2 transactions to produce ``withdrawable utxos'' that cannot be spent on L2 but can be withdrawn permissionlessly on L1.
This will allow L1 withdrawal orders to withdraw these L2 utxos without even requiring any signatures.}

A user initiates a withdrawal from Midgard by submitting an L1 transaction that performs the following:
\begin{enumerate}
    \item Spend an input \code{l1\_nonce}, which uniquely identifies this withdrawal order.
    \item Register a staking script credential to witness the withdrawal order.
      The staking script is parametrized by \code{l1\_id} (which is simply the hash of the serialized \code{l1\_nonce}), and the credential's purpose is to disprove the existence of the withdrawal order whenever the credential is \emph{not} registered.
    \item Mint a withdrawal order token to verify the following datum:
        \begin{equation*}
        \T{WithdrawalOrderDatum} \coloneq \left\{
            \begin{array}{ll}
                \T{event} : & \T{WithdrawalEvent}, \\
                \T{inclusion\_time} : & \T{PosixTime}, \\
                \T{witness} : & \T{ScriptHash}, \\
                \T{refund\_address}: & \T{Address}, \\
                \T{refund\_datum}: & \T{Option(Data)}
            \end{array}
            \right\}
        \end{equation*}
    \item Send min-ADA to the Midgard withdrawal order address, along with the withdrawal order token and the above datum.
\end{enumerate}

At the time of the L1 withdrawal order, its \code{inclusion\_time} is set to the sum of the L1 transaction's validity interval upper bound and the \code{event\_wait\_duration} Midgard protocol parameter.
According to Midgard's ledger rules:
\begin{description}
    \item[Withdrawal order inclusion.] A block header must include withdrawal orders with inclusion times falling within the block header's event interval, and it must \emph{not} include any other withdrawal orders.
\end{description}

The withdrawal order's outcome is determined as follows:
\begin{itemize}
    \item If the withdrawal event is included in a settlement queue node, then utxos from the Midgard reserve and confirmed deposits can be used to pay for the creation of an L1 utxo according to the withdrawal event.
    \item If the withdrawal order's inclusion time is within the confirmed header's event interval but not within the event interval of any settlement queue node, then the withdrawal order utxo can be refunded to its user according to the \code{refund\_address} and \code{refund\_datum} fields.
\end{itemize}

The withdrawal order's \code{witness} staking credential must be deregistered when the withdrawal order utxo is spent.

\subsection{Minting policy}
\label{h:withdrawal-order-minting-policy}

The \code{withdrawal} minting policy is statically parametrized on the \code{hub\_oracle} minting policy.
It oversees correctness of datums, and registration/unregistration of events' corresponding witness staking scripts.

\begin{description}
  \item[Authenticate Withdrawal.] Properly record a new withdrawal order event from L2 to L1.
    Conditions:
      \begin{enumerate}
        \item Let \code{l1\_nonce} be the output reference of a utxo on L1 that is spent in the order transaction.
        \item Let \code{l1\_id} be the Blake2b256 hash of serialized \code{l1\_nonce}.
        \item An NFT with own policy ID and asset name of \code{l1\_id} must be minted and included in the withdrawal order utxo.
        \item The witness staking script, instantiated with \code{l1\_id} must be registered in the transaction.
        \item The redeemer used for registering the witness script must be equal to the \code{withdrawal} policy ID.
        \item Let \code{withdrawal\_addr} be the address of the withdrawal contract from Midgard's hub oracle.
        \item Withdrawal order utxo must be produced at \code{withdrawal\_addr}.
        \item The withdrawal order's \code{inclusion\_time} must be equal to transaction's time-validity upper bound plus \code{event\_wait\_duration} Midgard protocol parameter.
        \item The hash of the witness script must be correctly stored in the withdrawal datum.
      \end{enumerate}
    \item[Burn Withdrawal NFT.] Oversee start of the funding for a withdrawal order by requiring its NFTs to be burnt, and having the utxo reproduced at the \code{escrow} contract.
    Conditions:
      \begin{enumerate}
        \item Let \code{l1\_id} be the corresponding ID of the target withdrawal order, provided via the redeemer.
        \item An NFT with own policy ID and asset name of \code{l1\_id} must be burnt.
        \item The witness staking script, instantiated with \code{l1\_id} must be unregistered in the transaction.
        \item The redeemer used for unregistrering the witness script must be correct.
          Namely, it must be equal to the \code{withdrawal} policy ID.
      \end{enumerate}
\end{description}

\subsection{Spending validator}
\label{h:withdrawal-order-spending-validator}

The \code{withdrawal} spending validator is statically parametrized on the \code{hub\_oracle} minting policy.
It's responsible for initializing the funding phase of a withdrawal order by reproducing its utxo at the \code{escrow} contract, or refunding stranded withdrawal orders.

\begin{description}
  \item[Transfer to Escrow.] Transaction for reproducing the order at the escrow contract to be filled up from the Midgard reserve.
    Conditions:
    \begin{enumerate}
      \item Let \code{settlement\_node} be the referenced settlement node.
      \item The withdrawal order must be included in the withdrawal tree of \code{settlement\_node}.
        This also implies the inclusion time of the order falls within the time interval of \code{settlement\_node}.
      \item Let \code{escrow\_addr} be the address of the intermediary escrow contract retrieved from hub oracle.
      \item The utxo produced at \code{escrow\_addr} must have the same \code{value} as the withdrawal order utxo, without the withdrawal NFT, and with the \code{escrow} NFT (same asset name) added.
      \item The \code{BurnEventNFT} endpoint of the withdrawal minting script must be invoked with the corresponding \code{l1\_id} of the withdrawal utxo.
      \item The minting logic from \code{escrow} must be invoked, minting an NFT with the same asset name as the withdrawal NFT being burnt in the transaction.
      \item The datum attached to the escrow utxo must hand over \code{l2\_value}, \code{l1\_address} and \code{l1\_datum} unchanged.
      \item No reference script must be attached to the escrow utxo.
    \end{enumerate}
  \item[Refund.] 
    Conditions:
    \begin{enumerate}
      \item The \code{BurnEventNFT} endpoint of the withdrawal minting script must be invoked with the corresponding \code{l1\_id} of the withdrawal order utxo.
      \item The min-ADA included in the withdrawal order utxo must go to the \code{refund\_address} specified in the withdrawal datum.
      \item The datum attached to the produced utxo must also match the one stated in withdrawal datum.
      \item 
        \begin{itemize}
          \item If inclusion time of the stranded withdrawal falls within the time range of an existing settlement node, refund can only be allowed if the settlement node's corresponding tree does not contain the withdrawal (proven by providing a non-membership proof).
          \item Otherwise, if the inclusion time falls in one of the time gaps of settlement queue, refund request is considered valid by referencing the immediate settlement node and showing the inclusion time falls in the gap.
        \end{itemize}
    \end{enumerate}
\end{description}

\subsection{Staking script}
\label{h:withdrawal-order-staking-script}

The \code{witness} script is parameterized by an ID that is obtained from the output reference of a specified utxo on L1, which is serialized and then hashed with Blake2b256.
The registration state of this script is the mechanism that allows related fraud proofs to function.
Conditions:

\begin{enumerate}
  \item Let \code{l1\_id} be the Blake2b256 hash of withdrawal event's \code{WithdrawalId} (L1 nonce output-reference).
  \item Let \code{withdrawal\_minting\_policy} be the policy ID of the withdrawal NFT, accessed via the redeemer.
  \item Registration/unregsitration of this witness script is respectively tied to the \code{withdrawal\_minting\_policy}'s mint/burn logic. 
    \begin{itemize}
      \item Registeration is allowed if a withdrawal NFT with asset name of \code{l1\_id} is minted.
      \item Unregistration is allowed if a withdrawal NFT with asset name of \code{l1\_id} is burnt.
    \end{itemize}
\end{enumerate}

\end{document}
