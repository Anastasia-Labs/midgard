\documentclass[../midgard.tex]{subfiles}
\graphicspath{{\subfix{../images/}}}
\begin{document}

\section{Transaction request (L2)}
\label{h:transaction-request}

A user's primary method of transacting on the Midgard ledger is to submit an L2 transaction directly to the current operator via a web-based API endpoint. Operators are expected to serve such APIs in a publicly accessible and employ sufficient security techniques to mitigate denial-of-service.

Users' L2 transactions follow the data structure defined in \cref{h:midgard-transaction-types}. In particular, users can freely set transaction time-validity intervals according to their preferences. As long as there is a non-empty overlap between a transaction's time-validity interval and a block's event interval, the transaction can be included in the block. Since event intervals are adjacent to each other, a user's valid transaction request can only miss the ledger in two cases:
\begin{itemize}
    \item The operator censors it (see mitigation in \cref{h:transaction-order}).
    \item The operator commits the last block overlapping with the transaction before receiving the transaction request.\footnote{Note that this case does not preclude transactions with short time-validity intervals from succeeding. For example, a transaction post-dated a couple of minutes in the future can still be included in a block even if its validity interval's duration is one millisecond.}
\end{itemize}

By contrast, L1 transactions on Cardano can fail in several other time-related ways. Ouroboros blocks only occupy every 20th one-second slot on average and have limited transaction capacity. This means that users often have to set longer transaction validity intervals than they want and wait for many block confirmations of their tx before relying on its outcomes.

However, while on Cardano L1 two transactions with non-overlapping validity intervals cannot be included in the same block, the analogous L2 transactions \emph{can} be included in a Midgard block if its event interval overlaps each transaction. Thus, transaction validity intervals on Midgard define only the relation between transactions and blocks but do not necessarily imply a temporal precedence relation between transactions.

\notebox{Which other information should we include here for L2 transaction requests?}\todo

\end{document}