\documentclass[../midgard.tex]{subfiles}
\graphicspath{{\subfix{../images/}}}
\begin{document}

\section{Escape hatch}
\label{h:escape-hatch}
\todo

An escape hatch block can be added to the state queue as follows:
\begin{enumerate}
    \item Spend transaction order and withdrawal order utxos, refunding all of their min-ADA.
    \item Reference compliance proof tokens verifying that the spent orders fully comply with Midgard's ledger rules.
    \item Append a node to the state queue with a special EscapeHatch datum.
\end{enumerate}

If the escape hatch block is removed (because it is a child of a fraudulent block), its transaction and withdrawal orders are lost.
However, they can be recreated, and their compliance proofs can be reused to create another escape hatch block (omitting any orders that are no longer valid).

An escape hatch block is merged into the confirmed state like an operator block: copy over the relevant fields to the ConfirmedHeader and append a node to the settlement queue if the escape hatch block contains any withdrawal events.

\end{document}
