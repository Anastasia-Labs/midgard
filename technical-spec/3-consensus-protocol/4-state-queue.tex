\documentclass[../midgard.tex]{subfiles}
\graphicspath{{\subfix{../images/}}}
\begin{document}

\section{State queue}
\label{h:state-queue}

The state queue is an L1 data structure that stores Midgard operators' committed block headers until they are confirmed.
Active operators from the operator directory (see \cref{h:operator-directory}) take turns committing block headers to the state queue according to the rotating schedule enforced by the scheduler (see \cref{h:scheduler}).

\subsection{Utxo representation}
\label{h:state-queue-utxo-representation}

The \code{state\_queue} is implemented as a key-unordered linked list of block headers (see \cref{h:block}).
Each \code{state\_queue} node's key is its block header's hash.
\begin{align*}
    \T{StateQueueDatum} &\coloneq \T{NodeDatum}(\T{Header}) \\ \\
    \T{valid\_key}(\T{scd} : \T{StateQueueDatum}) &\coloneq
        \Bigl( \T{scd.key} \equiv \T{Some}(\T{hash}(\T{scd.data})) \Bigr)
\end{align*}

Committing a block header to the \code{state\_queue} means appending a node containing the block header to the end of the queue.
After staying there for the \code{maturity\_duration} (a protocol parameter), it is merged to the confirmed state (held at the \code{state\_queue} root node) in the first-in-first-out (FIFO) order.

\subsection{Minting policy}
\label{h:state-queue-minting-policy}

The \code{state\_queue} minting policy controls the structural changes to the state queue.
It is statically parametrized on the \code{hub\_oracle}, \code{active\_operators}, \code{retired\_operators}, \code{scheduler}, and \code{fraud\_proof} minting policies.
Redeemers:

\begin{description}
    \item[Init.] Initialize the \code{state\_queue} via the Midgard hub oracle.
      Conditions:
        \begin{enumerate}
            \item The transaction must mint the Midgard hub oracle token.
            \item The transaction must Init the \code{state\_queue}.
        \end{enumerate}

    \item[Deinit.] Deinitialize the \code{state\_queue} via the Midgard hub oracle.
      Conditions:
        \begin{enumerate}
            \item The transaction must burn the Midgard hub oracle token.
            \item The transaction must Deinit the \code{state\_queue}.
        \end{enumerate}

    \item[Commit Block Header.] An operator commits a block header to the state queue if it is the operator's turn according to the rotating schedule.
      Grouped conditions:
        \begin{itemize}
            \item Commit the block header to the state queue, with the operator's consent:
            \begin{enumerate}
                \item Let \code{operator} be a redeemer argument indicating the key of the operator committing the block header.
                \item The transaction must be signed by \code{operator}.
                \item The transaction must Append a node to the \code{state\_queue}.
                  Among the transaction outputs, let that node be \code{header\_node} and its predecessor node be \code{previous\_header\_node}.
                \item \code{operator} must be the operator of \code{header\_node}.
                \item Let \code{previous\_operator} be the operator of \code{previous\_header\_node}.
                \item The \code{key} field of \code{header\_node} must be the hash of its \code{data} field.
            \end{enumerate}

            \item Verify that it is the operator's turn to commit according to the scheduler:
            \begin{enumerate}[resume]
                \item The transaction must include a reference input with the scheduler NFT.
                  Let that input be the \code{scheduler\_state}.
                \item The \code{operator} field of \code{scheduler\_state} must match \code{operator}.
            \end{enumerate}

            \item Verify block timestamps:
            \begin{enumerate}[resume]
                \item The \code{start\_time} of \code{header\_node} must be equal to the \code{end\_time} of the \code{previous\_header\_node}.
                \item The \code{end\_time} of \code{header\_node} must match the transaction's time-validity upper bound.
                \item The \code{end\_time} of \code{header\_node} must be within the shift interval of \code{scheduler\_state}.
            \end{enumerate}
            
            \item Update the operator's timestamp in the active operators set:
            \begin{enumerate}[resume]
                \item The transaction must include an input, spent via the Update Commitment Time redeemer, of an \code{active\_operators} node with a key matching the \code{operator}.
                  Let \code{operator\_node} be that node.
            \end{enumerate}
        \end{itemize}

    \item[Merge To Confirmed State.] If a block header is mature, merge it to the confirmed state of the \code{state\_queue}.
      Conditions:
    \begin{enumerate}
        \item The transaction must Remove a node from the \code{state\_queue}.
          Let \code{header\_node} be the removed node and \code{confirmed\_state\_node} be its predecessor node before removal.
        \item \code{confirmed\_state\_node} must be the root node of \code{state\_queue}.
        \item \code{header\_node} must be mature --- the lower bound of the transaction validity interval meets or exceeds the sum of the \code{commitment\_time} field of \code{header\_node} and the Midgard \code{maturity\_duration} protocol parameter.
        \item \code{header\_node} and \code{confirmed\_state\_node} must match on the \code{data} field.
    \end{enumerate}

    \item[Remove Fraudulent Block Header.] Remove a fraudulent block header from the state queue and slash its operator's ADA bond.
      Grouped conditions:
    \begin{itemize}
        \item Remove the fraudulent block header:
        \begin{enumerate}
            \item Let \code{fraudulent\_operator} be a redeemer argument indicating the operator who committed the fraudulent block header.
            \item The transaction must Remove a node from the \code{state\_queue}.
              Let \code{removed\_node} be the removed node and \code{predecessor\_node} be its predecessor node before removal.
            \item \code{fraudulent\_operator} must match the key of \code{removed\_node}.
        \end{enumerate}

        \item Slash the fraudulent operator:
        \begin{enumerate}[resume]
            \item Let \code{operator\_status} be a redeemer argument indicating whether the fraudulent operator is active or retired.
            \item If \code{operator\_status} is active:
                \begin{enumerate}
                    \item The transaction must Remove a node from the \code{active\_operators} set via the Remove Operator Slash Bond redeemer.
                      The \code{slashed\_operator} argument provided to that redeemer must match \code{fraudulent\_operator}.
                \end{enumerate}
            \item Otherwise:
                \begin{enumerate}
                    \item The transaction must Remove a node from the \code{retired\_operators} set via the Remove Operator Slash Bond redeemer.
                      The \code{slashed\_operator} argument provided to that redeemer must match \code{fraudulent\_operator}.
                \end{enumerate}
        \end{enumerate}

        \item Verify that fraud has been proved for the removed node or its predecessor:
        \begin{enumerate}[resume]
            \item The transaction must include a reference input holding a \code{fraud\_proof} token.
            \item Let \code{fraud\_proof\_block\_hash} be the last 28 bytes of the \code{fraud\_proof} token.
            \item One of the following must be true:
                \begin{enumerate}
                    \item \code{fraud\_proof\_block\_hash} matches the \code{predecessor\_node} key.
                      This means that a child of the fraudulent node is being removed.
                    \item \code{fraud\_proof\_block\_hash} matches the \code{removed\_node} key, and last node of \code{state\_queue} is \code{removed\_node}.
                      This means that the fraudulent node is being removed and has no more children.
                \end{enumerate}
        \end{enumerate}
    \end{itemize}
\end{description}

\notebox{Add redeemers for escape hatch blocks.}\todo

\subsection{Spending validator}
\label{h:state-queue-spending-validator}

The spending validator of \code{state\_queue\_addr} always forwards to its corresponding minting policy (statically parametrized) and requires the transaction to invoke it.
It does not allow any in-place modifications to the \code{data} field of nodes in \code{state\_queue}.
Conditions:
\begin{enumerate}
    \item The transaction must mint or burn tokens of the \code{state\_queue} minting policy.
\end{enumerate}

\end{document}
