\documentclass[../midgard.tex]{subfiles}
\graphicspath{{\subfix{../images/}}}
\begin{document}

\section{Fraud proof computation threads}
\label{h:fraud-proof-computation-threads}

A computation thread splits up a large computation into a series of steps, passing control between the steps in the continuation-passing style (CPS).
In other words, the computation thread is a state machine (see \cref{h:single-threaded-state-machine}) with a linear state graph.

Each step is a spending validator that executes the following sequence:
\begin{enumerate}
    \item Parses the computation state from its datum.
    \item Optionally receives arguments from its redeemer to guide the computation step.
    \item Advances the computation by the step, producing a new computation state.
    \item Suspends the computation by serializing the new computation state into a new datum that it sends to the spending validator of the next step.
\end{enumerate}

All steps in a computation thread parametrize the same datum type:
\begin{equation*}
    \T{StepDatum} (\T{state\_data}) \coloneq \left\{
    \begin{array}{ll}
        \T{fraud\_prover}  : & \T{PubKeyHash} \\
        \T{data} : & \T{state\_data}
    \end{array} \right\}
\end{equation*}

\subsection{Minting policy}
\label{h:fraud-proof-computation-threads-minting-policy}

The \code{computation\_thread} minting policy initializes its state machine.
It is statically parametrized on the \code{fraud\_proof\_catalogue} and \code{hub\_oracle} minting policies.
Redeemers:

\begin{description}
    \item[Init.] Conditions:
        \begin{enumerate}
            \item The transaction must reference a node of the \code{fraud\_proof\_catalogue}.
              Let that node be \code{fraud\_category}.
            \item The transaction must include the Midgard hub oracle NFT in a reference input.
            \item Let \code{state\_queue} be the policy ID in the corresponding field of the Midgard hub oracle.
            \item The transaction must reference a node from the \code{state\_queue}.
              Let \code{fraud\_node} be that node.
            \item The transaction must mint a single token of the \code{computation\_thread} minting policy.
              The token name must concatenate the four-byte key of \code{fraud\_category} and the 28-byte block header hash in the key of \code{fraud\_node}.
            \item The computation thread token must be sent to the spending validator defined in the \code{init\_step} field of \code{fraud\_category}.
              Let \code{output\_state} be that transaction output.
            \item The \code{output\_state} datum type must be \code{StepDatum(Void)}.\footnote{Aiken calls it StepDatum(Void), while Plutarch calls it StepDatum(PUnit).} 
            \item The transaction must be signed by the \code{fraud\_prover} pub-key hash of \code{output\_state}.
            \item Other than ADA, \code{output\_state} must \emph{not} hold any other tokens.
            \item The transaction must \emph{not} mint or burn any other tokens.
        \end{enumerate}
    \item[Success.] Terminate the state machine normally from the final spending validator in the computation.
      There are no conditions because this redeemer relies on the spending validator to burn the token.
    \item[Cancel.] Terminate the state machine exceptionally from any spending validator in the computation.
      There are no conditions because this redeemer relies on the spending validator to burn the token.
\end{description}

\subsection{Spending validators}
\label{h:fraud-proof-computation-threads-spending-validators}

A fraud-proof computation succeeds if its thread token passes through all the steps' spending validator addresses.
In that case, the last step's spending validator reifies the successful fraud-proof by requiring a fraud-proof token to be minted.
The fraud-proof minting policy requires the computation thread token to be burned, specifically via the Success redeemer.

On the other hand, at any step, the person who initiated the computation thread can cancel the computation instead of advancing it.
In that case, the step's spending validator requires the computation thread token to be burned via the Cancel redeemer.
Thus, while there is typically only one path for a computation thread to reach success via the sequential steps,\footnote{Technically, if there are multiple instances of the same fraud category in a fraudulent block, then there is a corresponding number of paths to prove the occurrence of that fraud category in the block.
The redeemer arguments provided to the computation steps collectively select one of these paths.} there may be multiple opportunities for the computation to be canceled along the way.

For each fraud-proof category, each spending validator is custom-written to express the specific logic of that computation step, and it is statically parametrized on the next step's spending validator (if any).
One of the custom conditions of the computation step should verify the transition between the input state and output state of the thread:
\begin{align*}
    \T{verify\_transition} &: (\T{Input}, \T{Output}, ..\T{Args}) -> \T{Bool} \\
    \T{verify\_transition(i, o, ..args)} &\coloneq
        \Bigl( \T{transition(i, ..args) \equiv \T{o}} \Bigr) \\
    \T{transition} &: (\T{Input}, ..\T{Args}) -> \T{Output}
\end{align*}

All of the spending validators share the same parametric redeemer type, but each spending validator can parametrize the Continue redeemer by a different type to hold the custom instructions needed to guide the computation step:
\begin{equation*}
    \T{StepRedeemer} (\T{instructions}) \coloneq
        \T{Continue}(\T{instructions}) \;|\;
        \T{Cancel}
\end{equation*}

These redeemers should be handled in the following general pattern:
\begin{description}
    \item[Continue.] Advance the computation.
      Conditions:
        \begin{enumerate}
            \item If this is the last step of the computation:
                \begin{itemize}
                    \item Mint the fraud token, which will implicitly burn the computation thread token with the Success redeemer.
                      Let \code{output\_state} be that transaction output.
                    \item The \code{output\_state} datum type must be \code{StepDatum(Void)}.
                    \item The \code{fraud\_prover} field must match between the \code{output\_state} and the input datum.
                \end{itemize}
            \item Otherwise:
                \begin{itemize}
                    \item The computation thread token must be sent to the next step's spending validator.
                      Let \code{output\_state} be that transaction output.
                    \item The \code{fraud\_prover} field must match between the \code{output\_state} and the input datum.
                \end{itemize}
            \item Evaluate the custom conditions of the computation step, including verifying the state transition.
            \item The custom conditions may require the transaction to reference a state queue with a key hash matching the last 28 bytes of the computation thread token name.
            \item The transaction must \emph{not} mint or burn any other tokens.
        \end{enumerate}
    \item[Cancel.] Cancel the computation.
      Conditions:
        \begin{enumerate}
            \item Burn the computation thread token with the Cancel redeemer.
            \item Return the ADA from the computation thread utxo to the fraud prover pub-key defined in the input datum.
            \item The transaction must \emph{not} mint or burn any other tokens.
        \end{enumerate}
\end{description}

\end{document}
