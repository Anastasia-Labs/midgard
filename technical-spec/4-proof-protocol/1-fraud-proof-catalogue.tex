\documentclass[../midgard.tex]{subfiles}
\graphicspath{{\subfix{../images/}}}
\begin{document}

\section{Fraud proof catalogue}
\label{h:fraud-proof-catalogue}

Midgard uses a key-unordered linked list (see \cref{h:key-ordered-list}) to track all possible categories of fraud that can occur in Midgard blocks.
Every enforced Midgard ledger rule (see \cref{h:ledger-rules-and-fraud-proofs}) must be represented by a fraud proof category indicating the violation of that rule.

Each node in the \code{fraud\_proof\_catalogue} specifies the script hash of the first step in a fraud proof computation (see \cref{h:fraud-proof-computation-threads}), which inspects a given block and succeeds if the block has an instance of the fraud category.
The node keys are sequentially ordered and 4 bytes each, allowing the catalogue to track 4096 fraud proof categories.

\begin{align*}
    \T{FraudProofCatalogueDatum} &\coloneq \left\{
    \begin{array}{ll}
        \T{init\_step}  : & \T{ScriptHash}
    \end{array} \right\} \\
\end{align*}

\subsection{Minting policy}
\label{h:fraud-proof-catalogue-minting-policy}

The \code{fraud\_proof\_catalogue} minting policy is statically parametrized on the \code{hub\_oracle} minting policy and the Midgard governance key.\footnote{For now, Midgard governance is centralized on a pubkey.
This should be replaced by the proper governance mechanism when it is specified.} Redeemers:
\begin{description}
    \item[Init.] Initialize the \code{fraud\_proof\_catalogue} via the Midgard hub oracle.
      Conditions:
        \begin{enumerate}
            \item The transaction must mint the Midgard hub oracle NFT.
            \item The transaction must Init the \code{fraud\_proof\_catalogue} list.
        \end{enumerate}
    \item[Deinit.] Deinitialize the \code{fraud\_proof\_catalogue} via the Midgard hub oracle.
      Conditions:
        \begin{enumerate}
            \item The transaction must burn the Midgard hub oracle NFT.
            \item The transaction must Deinit the \code{fraud\_proof\_catalogue} list.
        \end{enumerate}
    \item[New Fraud Category.] Catalogue a new Midgard fraud category.
      Conditions:
        \begin{enumerate}
            \item The transaction must Append a node to the \code{fraud\_proof\_catalogue}.
              Let that node be \code{new\_node}.
            \item The Midgard governance key must sign the transaction.
            \item Let \code{old\_node} be the node that links to \code{new\_node} in the transaction outputs.
            \item The \code{new\_node} key must be four bytes long.
            \item If \code{old\_node} is the root node:
                \begin{itemize}
                    \item The \code{new\_node} key must be zero.
                \end{itemize}
            \item Otherwise:
                \begin{itemize}
                    \item The \code{old\_node} key must be less than 4095.
                    \item The \code{new\_node} must be larger than \code{old\_node} key by one.
                \end{itemize}
        \end{enumerate}
    \item[Remove Fraud Category.] Remove a Midgard fraud category.
      Conditions:
        \begin{enumerate}
            \item The transaction must Remove a node from the \code{fraud\_proof\_catalogue}.
            \item The Midgard governance key must sign the transaction.
        \end{enumerate}
\end{description}

\subsection{Spending validator}
\label{h:fraud-proof-catalogue-spending-validator}

The spending validator of \code{fraud\_proof\_catalogue\_addr} is statically parametrized on the \code{fraud\_proof\_catalogue} minting policy.
Conditions:
\begin{enumerate}
    \item The transaction must burn a \code{fraud\_proof\_catalogue} token.
    \item The transaction must \emph{not} mint or burn any other tokens.
\end{enumerate}

\end{document}
