\documentclass[../midgard.tex]{subfiles}
\graphicspath{{\subfix{../images/}}}
\begin{document}

\section*{Introduction}
\label{h:introduction}

Midgard is a Layer 2 (L2) scaling solution for the Cardano blockchain. It employs optimistic rollup technology to enhance Cardano's capacity to process transactions and host more complex applications, delivering a richer user experience at a more competitive cost. As Cardano continues to grow in usage and demand, scaling solutions like Midgard are critical for maintaining high performance and low transaction costs. This whitepaper describes the architecture and technical design of the Midgard protocol, detailing how it integrates with Cardano's Layer 1 (L1) to deliver secure and efficient transaction processing.

Optimistic rollups process blocks of transactions off-chain and commit the hash summaries of those blocks onchain to the L1 ledger. Each block is committed by a Midgard operator who guarantees the block's validity. The block then waits in a queue for at least a fixed duration to be merged into the confirmed state of the optimistic rollup on L1. The operator must collateralize their guarantee with a bond deposit and publish the full contents of the block on the publicly accessible data availability (DA) layer.

While a committed block is queued, anyone can inspect its contents on the data availability layer and ascertain whether it is valid. If someone detects that the block is invalid, they can submit a fraud proof to prevent it from being merged into the confirmed state, slash the operator's bond, and receive a portion of the forfeited bond as a reward. Thus, an optimistic rollup can process a large number of transactions offchain while maintaining security and finality properties that are similar to transactions processed directly onchain, as long as:
\begin{itemize}
    \item The bond requirement for the rollup's operators is large enough to discourage fraud.
    \item The reward for preventing an invalid block from merging is large enough to encourage public vigilance in watching the operators.
    \item The waiting period for committed blocks is long enough to allow the watchers to detect and prove fraud before those blocks are merged.
    \item The data availability layer is accessible by anyone who wishes to inspect the rollup blocks at any time that they wish to do so.
\end{itemize}

Whenever the latter three security parameter values are calibrated to provide a high probability of invalid blocks being detected and disqualified, the bond requirement is a strong deterrent against operators attempting fraud. An operator cannot dismiss the forfeited bond as merely a "cost of doing business" paid to obtain potentially larger revenues from fraud. Whenever an invalid block is disqualified, it does \emph{not} affect the confirmed state, so there are no revenues from that fraud to offset the operator's forfeited bond.

The main design goal of Midgard is to streamline the processes by which blocks are committed/merged, fraud is detected, and fraud proofs are verified onchain. Advancing this goal allows the security parameters to be calibrated to achieve a better balance between security, transaction throughput, confirmation time, and community participation in committing blocks and detecting fraud.

\subsection*{Scalability and efficiency}
\label{h:scalability-and-efficiency}

By processing transactions off-chain and only validating them on-chain when fraud proofs challenge them, Midgard significantly increases throughput and reduces costs for Cardano transactions. Its rollup blocks use sparse Merkle trees and compact state representations to enhance the protocol's efficiency further, enabling it to handle a large volume of transactions without overburdening the L1.

The deterministic nature of Cardano transactions allows Midgard fraud proofs to pinpoint the specific site of a transaction that violated Midgard's ledger rules, without having to look at any other parts of that transaction, any other unrelated transactions within the block, or any other blocks. This keeps fraud proofs and their onchain validation procedures small and efficient, which reduces the time and cost needed to submit fraud proofs when invalid blocks are detected, which makes it feasible for a wider group of people to police Midgard's blocks. In this way, Midgard significantly reduces fraud proof size relative to optimistic rollups used in Ethereum and other account-based blockchain ecosystems, where a much larger part of the global blockchain state needs to be inspected when constructing and verifying a fraud proof.

\subsection*{Fallback mechanisms on L1}

\notebox{Midgard's L1 architecture includes the deposit, withdrawal, mailbox, and escape-hatch contracts to allow users to force deposits and withdrawal requests to be processed within a reasonable timeframe, even if all operators stop participating in the protocol.}\todo

\subsection*{Feedback and contributions welcome!}
\label{h:feedback-and-contributions-welcome}

This document specifies the ledger, onchain contracts, and offchain components of the Midgard architecture. These components interact to form a cohesive system that scales Cardano while preserving its security guarantees. We detail the roles of users, operators, and watchers, how blocks are committed and merged on L1, and how fraud proofs are handled.

We invite developers, validators, and the broader community to engage with this document and offer feedback and contributions to help refine and enhance the protocol.

\end{document}