\documentclass[../midgard.tex]{subfiles}
\graphicspath{{\subfix{../images/}}}
\usepackage{amsmath}
\begin{document}

\section{Midgard Ledger Rules and Fraud Proofs}
\label{h:ledger-rules-fraud-proofs}

In the following sections the following premises are used:

\begin{equation*}
\begin{split}
           b & \in Blocks          \\
         txs & := transactions(b)  \\
 utxos_{pre} & := prev\_utxos(b)   \\
utxos_{post} & := utxos(b)         \\
        wtxs & := withdrawals(b)
\end{split}
\end{equation*}

\subsection{All inputs must be valid}
\label{rule:all-inputs-must-be-valid}

\begin{description}

\item[Formal specification:]
\begin{equation*}
\begin{split}
    \forall t \in Ledger, &\forall i \in spend\_inputs(t): \\
    &( \exists t_1 \in Ledger, t \neq t_1 \land i \in outputs(t_1) ) \land \\
    &( \nexists t_2 \in Ledger, t \neq t_2 \land i \in spend\_inputs(t_2) )
\end{split}
\end{equation*}

\item[Description:] A transaction cannot spend a non-existing (or an already spent) UTxO.
\item[Violating conditions:] If any of the following conditions are met (\hyperref[violation:NO-INPUT]{NO-INPUT}, \hyperref[violation:INPUT-NO-IDX]{INPUT-NO-IDX}, \hyperref[violation:WITHDRAWN-INPUT]{WITHDRAWN-INPUT}, \hyperref[violation:DOUBLE-SPEND]{DOUBLE-SPEND} or \hyperref[violation:DOUBLE-WITHDRAW]{DOUBLE-WITHDRAW}), then this property is broken, the new block must be eliminated.

\end{description}

\subsubsection{NO-INPUT violation}
\label{violation:NO-INPUT}

\begin{description}

\item[Formal specification:]
\begin{equation*}
\begin{split}
    \exists t \in txs, &\exists i \in spend\_inputs(t): \\
    &( i \notin utxos_{prev} ) \land  \\
    &( \nexists t_1 \in txs, t \neq t_1 \land tx\_hash(t_1) = tx\_hash(i) )
\end{split}
\end{equation*}

\item[Fraud proof construction:] In order to prove that there exist transactions $t$ that violates this rule:
\begin{itemize}
    \item a membership proof must be provided for $t$ that shows that the transaction is included in the block ($t \in txs$)
    \item a membership proof for input $i$ must be specified such that $i \in spend\_inputs(t)$
    \item a non-membership proof must be created to show that $i$ is not in $utxos_{prev}$
    \item a non-membership proof must also be generated that shows that there are no transactions in $txs$ that have id $tx\_hash(i)$
\end{itemize}

\item[Rationale:] The items defined above together prove that transaction $t$ attempted to spend the UTxO $i$ that does not exist or was spent in a previous block.

\end{description}

\subsubsection{INPUT-NO-IDX violation}
\label{violation:INPUT-NO-IDX}

\begin{description}

\item[Formal specification:]
\begin{equation*}
\begin{split}
    \exists t \in txs, &\exists i \in spend\_inputs(t): \\
    &( \exists t_1 \in txs, tx\_hash(t_1) = tx\_hash(i) \land i \notin outputs(t_1) )
\end{split}
\end{equation*}

\item[Fraud proof construction:] In order to prove that there exist transactions $t$ that violates this rule:
\begin{itemize}
    \item a membership proof must be provided for $t$ that shows that the transaction is included in the block ($t \in txs$)
    \item a membership proof must be presented for input $i$ such that $i \in spend\_inputs(t)$
    \item a membership proof must be created for $t_1$ such that $tx\_hash(t_1) = tx\_hash(i)$
    \item a DA layer proof must be presented that certifies that $length(outputs(t_1)) < index(i)$ \todo
\end{itemize}

\item[Rationale:] The items defined above prove that $t$ attempted to spend the input $i$, which did not exist at all.

\end{description}

\subsubsection{WITHDRAWN-INPUT violation}
\label{violation:WITHDRAWN-INPUT}

\begin{description}

\item[Formal specification:]
\begin{equation*}
    \exists t \in txs, \exists i \in spend\_inputs(t), \exists w \in wtxs, i = l2\_outref(wtx)
\end{equation*}

\item[Fraud proof construction:] In order to prove that there exist transactions $t$ that violates this rule:
\begin{itemize}
    \item a membership proof must be provided for $t$ that shows that the transaction is included in the block ($t \in txs$)
    \item a membership proof must be generated for input $i$ such that $i \in spend\_inputs(t)$
    \item a membership proof must be created to show that $w$ is in $wtxs$, which also spends input $i$
\end{itemize}

\item[Rationale:] The items defined above prove that $t$ attempted to spend the input $i$, which was spent in a withdraw transaction.

\end{description}

\subsubsection{DOUBLE-SPEND violation}
\label{violation:DOUBLE-SPEND}

\begin{description}

\item[Formal specification:]
\begin{equation*}
    \exists t \in txs, \exists i \in spend\_inputs(t), \exists t_1 \in tx, t \neq t_1 \land i \in spend\_inputs(t_1)
\end{equation*}

\item[Fraud proof construction:] In order to prove that there exist transactions $t$ that violates this rule:
\begin{itemize}
    \item a membership proof must be provided for $t$ that shows that the transaction is included in the block ($t \in txs$)
    \item a membership proof must be generated for input $i$ such that $i \in spend\_inputs(t)$
    \item a membership proof must be created to show that $t_1$ is in $txs$
    \item a membership proof must be given that verifies that $i \in spend\_inputs(t_1)$
\end{itemize}

\item[Rationale:] The items defined above prove that $t$ attempted to spend the input $i$, which was spent in another transaction.

\end{description}

\subsubsection{DOUBLE-WITHDRAW violation}
\label{violation:DOUBLE-WITHDRAW}

\begin{description}

\item[Formal specification:]
\begin{equation*}
    \exists w, w_1 \in wtxs, w \neq w_1 \land l2\_outref(w) = l2\_outref(w_1)
\end{equation*}

\item[Fraud proof construction:] In order to prove that there exist transactions $t$ that violates this rule:
\begin{itemize}
    \item a membership proof must be provided for $w$ that shows that the withdraw transaction is included in the block ($w \in wtxs$)
    \item a membership proof must be created to show that $w_1$ is in $wtxs$
\end{itemize}

\item[Rationale:] The items defined above prove that $w$ attempted to spend the same input, which was already withdrawn by $w_1$.

\end{description}

\subsection{Transaction validity range}
\label{rule:transaction-validity-range}

\begin{description}

\item[Formal specification:]
\todo
\begin{equation*}
\begin{split}
    \forall t \in Ledger, time\_range(block(t)) \subseteq validity\_interval(t)
\end{split}
\end{equation*}

\item[Description:] Every valid transaction in the ledger must be included at a timestamp that conforms to the validity range that the transaction prescribes.
  Whenever the condition \hyperref[violation:INVALID-RANGE]{INVALID-RANGE} is met, that indicates that this property is violated, and the block in question must not be included in the $Ledger$.

\end{description}

\subsubsection{INVALID-RANGE violation}
\label{violation:INVALID-RANGE}

\begin{description}

\item[Violating condition:]
\begin{equation*}
    \exists t \in txs, time\_range(b) \nsubseteq validity\_interval(t)
\end{equation*}

\item[Fraud proof construction:] In order to prove that there exists a transaction $t$ that violates this rule:
\begin{itemize}
    \item a membership proof must be provided that proves that transaction $t$ is included in the block $b$
\end{itemize}

\item[Rationale:] with the provided proofs, it can be shown that $t$ is included in the block $b$ with the violating condition $time\_range(b) \nsubseteq validity\_range(t)$ being true.

\end{description}

\subsection{At least one input}
\label{rule:at-least-one-input}

\begin{description}

\item[Formal specification:]
\begin{equation*}
    \forall t \in Ledger, |spend\_inputs(t)| > 0
\end{equation*}

\item[Description:] Every valid transaction in the ledger must spend at least one UTxO.
  If the violating condition \hyperref[violation:ZERO-INPUT]{ZERO-INPUT} is satisfied, then this property is broken and the block must be removed.

\end{description}

\subsubsection{ZERO-INPUT violation}
\label{violation:ZERO-INPUT}

\begin{description}

\item[Formal specification:]
\begin{equation*}
    \exists t \in txs, |spend\_inputs(t)| = 0
\end{equation*}

\item[Fraud proof construction:] In order to prove that there exists a transaction $t$ that violates this rule:
\begin{itemize}
    \item a membership proof must be provided that shows that the transaction is included in the ledger
    \item a DA layer proof must be presented that certifies that $length(spend\_inputs(t)) = 0$ \todo
\end{itemize}

\item[Rationale:] with the provided proof, it can be shown that a transaction $t$ is in the ledger, while $|inputs(t)| = 0$

\end{description}

\subsection{Minimum fee}
\label{rule:minimum-fee}

\begin{description}

\item[Formal specification:]
\begin{equation*}
    \forall t \in Ledger, tx\_fee(t) \geq min\_fee(t)
\end{equation*}

\item[Description:] Every valid transaction in the ledger must pay the fees for inclusion.
  Proving violating condition \hyperref[violation:MIN-FEE]{MIN-FEE} indicates that this property is broken.

\item[Calculating the minimum fee:] The fee calculation algorithm is the same as in Cardano ... \todo

\end{description}

\subsubsection{MIN-FEE violation}
\label{violation:MIN-FEE}

\begin{description}

\item[Formal specification:]
\begin{equation*}
    \exists t \in txs, fee(t) < min\_fee(t)
\end{equation*}

\item[Fraud proof construction:] In order to prove that there exists a transaction $t$ that violates this rule:
\begin{itemize}
    \item a membership proof must be provided that shows that the transaction is included in the ledger
\end{itemize}

\item[Rationale:] with the provided proof, it can be shown that a transaction $t$ is in the ledger, while $fee(t) < min\_fee(t)$

\end{description}

\subsection{Required signatures are correct}
\label{rule:required-signatures-are-correct}

\begin{description}

\item[Formal specification:]
\begin{equation*}
\begin{split}
    \forall t & \in Ledger, required\_signer\_hashes(t) = \\
    & \{ \, paymentHK(addr(u)) \, | \, (r, u) \in utxos_{post}, r \in spend\_inputs(t), addr(u) \in Addr^{vkey} \, \}
\end{split}
\end{equation*}
    
\item[Description:] Every valid transaction in the ledger must correctly show the required signers.
  Proving the violating condition \hyperref[violation:MISSING-REQ-SIGNER]{MISSING-REQ-SIGNER} or \hyperref[violation:NON-REQ-SIGNER]{NON-REQ-SIGNER} testifies that this property is broken and the block must not be included in the ledger.
    
\end{description}

\subsubsection{MISSING-REQ-SIGNER violation}
\label{violation:MISSING-REQ-SIGNER}

\begin{description}

\item[Formal specification:]
\begin{equation*}
\begin{split}
    \exists t & \in txs, \exists r \in spend\_inputs(t), \exists u \in Output, \\
    & (r, u) \in utxos_{post} \land paymentHK(addr(u)) \notin required\_signer\_hashes(t) \, \land \\
    & addr(u) \in Addr^{vkey}
\end{split}
\end{equation*}

\item[Fraud proof construction:] In order to prove that there exists a transaction $t$ that violates this rule:
\begin{itemize}
    \item a membership proof must be provided that shows that the transaction is included in the ledger
    \item a DA layer proof must be presented that shows that $r \in spend\_inputs(t)$
    \item a membership proof must be generated that shows that $(r, u) \in utxos_{post}$
    \item a DA layer proof must be shown that proves that \\ $paymentHK(addr(u)) \notin required\_signer\_hashes(t)$
\end{itemize}

\item[Rationale:] with the provided proof, it can be shown that a transaction $t$ is in the ledger and it violates the "Required signatures" property.

\end{description}

\subsubsection{NON-REQ-SIGNER violation}
\label{violation:NON-REQ-SIGNER}

\begin{description}

\item[Formal specification:]
\begin{equation*}
\begin{split}
    \exists t & \in txs, \exists vkey \in required\_signer\_hashes(t), \nexists (r, u) \in utxos_{post}, \\
    & r \in spend\_inputs(t) \land paymentHK(addr(u)) = vkey \land addr(u) \in Addr^{vkey}
\end{split}
\end{equation*}

\item[Fraud proof construction:] In order to prove that there exists a transaction $t$ that violates this rule:
\begin{itemize}
    \item a membership proof must be provided that shows that the transaction $t$ is included in the ledger
    \item a DA layer proof must be presented that shows that for a specified $vkey$ there is no utxo $u \in utxos_{post}$, such that the address of $u$ corresponds to $vkey$ and $t$ spends $u$
\end{itemize}

\item[Rationale:] with the provided proof, it can be shown that a transaction $t$ is in the ledger and it violates the "Required signatures" property.

\end{description}

\subsection{Signatures are valid}
\label{rule:signatures-are-valid}

\begin{description}

\item[Formal specification:]
\begin{equation*}
    \forall t \in Ledger, \forall (v, s, h) \in addr\_tx\_wits(t), is\_valid\_signature(v, s, h)
\end{equation*}
        
\item[Description:] Every provided signature must be valid.
  Proving the violating condition indicates that this property is broken, and thus the block must not be allowed in the ledger.
        
\end{description}

\subsubsection{INVALID-SIGNATURE violation}
\label{violation:INVALID-SIGNATURE}

\begin{description}

\item[Formal specification:]
\begin{equation*}
    \exists t \in txs, \exists (v, s, h) \in addr\_tx\_wits(t), \lnot is\_valid\_signature(v, s, h)
\end{equation*}

\item[Fraud proof construction:] In order to prove that there exists a transaction $t$ that violates this rule:
\begin{itemize}
    \item a membership proof must be provided that shows that the transaction $t$ is included in the ledger
    \item a membership proof must be shown that states that $(v, s, h) \in addr\_tx\_wits(t)$
\end{itemize}

\item[Rationale:] with the provided proof it can be shown that there exists an invalid signature for transaction $t$ (if indeed the signature $(v, s, h)$ is not valid).

\end{description}

\subsection{Every needed signature is provided}
\label{rule:every-needed-signature-is-provided}

\begin{description}

\item[Formal specification:]
\begin{equation*}
    \forall t \in Ledger, \forall h \in required\_signer\_hashes(t), \exists (v, s, h) \in addr\_tx\_wits(t)
\end{equation*}
        
\item[Description:] Every requires signature is provided.
  Proving the violating condition indicates that this property is broken, and thus the block must not be allowed in the ledger.

\end{description}

\subsubsection{MISSING-SIGNATURE violation}
\label{violation:MISSING-SIGNATURE}

\begin{description}

\item[Formal specification:]
\begin{equation*}
    \exists t \in txs, \exists h \in required\_signer\_hashes(t), h \notin \{ h_p \, | \, (v, s, h_p) \in addr\_tx\_wits(t) \}
\end{equation*}

\item[Fraud proof construction:] In order to prove that there exists a transaction $t$ that violates this rule:
\begin{itemize}
    \item a membership proof must be provided that shows that the transaction $t$ is included in the ledger
    \item a DA layer proof must be shown that states that $h \in required\_signer\_hashes(t)$
    \item a DA layer proof must be presented that shows that a signature with $h$ does not exist
\end{itemize}

\item[Rationale:] with the provided proof it can be shown that a required signature, corresponding to $h$ is missing.

\end{description}

\subsection{Multisig scripts are available}
\label{rule:multisig-scripts-are-available}

\subsection{Multisig scripts validated}
\label{rule:multisig-scripts-validated}

\subsection{Value preservation}
\label{rule:value-preservation}

\begin{description}

\item[Formal specification:]
\begin{equation*}
\begin{split}
    \forall t & \in Ledger, \\
    & mint(t) + \sum_{i \, \in \, spend\_inputs(t)} value(i) = fee(t) + \sum_{o \, \in \, outputs(t)} value(o)
\end{split}
\end{equation*}
        
\item[Description:] The total value must be preserved.
  Proving the violating condition indicates that this property is broken, and thus the block must not be allowed in the ledger.

\end{description}

% !!! TODO: violating conditions

\subsection{No Ada minted}
\label{rule:no-ada-minted}

\begin{description}

\item[Formal specification:]
\begin{equation*}
    \forall t \in Ledger, lovelaces(mint(t)) = 0
\end{equation*}
            
\item[Description:] Ada must not be minted.
  Proving the violating condition indicates that this property is broken, and thus the block must not be allowed in the ledger.
    
\end{description}

\subsection{Maximum transaction size}
\label{rule:maximum-transaction-size}

!!! TODO: is this relevant?

\subsection{Maximum reference script size}
\label{rule:maximum-reference-script-size}

!!! TODO: This is a placeholder, it will only be relevant once Midgard implements phase 2 validations (smart contracts).

\subsection{Maximum value size}
\label{rule:maximum-value-size}

!!! TODO: ...

\subsection{No negative value}
\label{rule:no-negative-value}

\begin{description}

\item[Formal specification:]
\begin{equation*}
    \forall t \in Ledger, mint(t) \geq \mathbf{0} \, \land fee(t) \geq 0 \land \forall o \in outputs(t), \, value(o) \geq \mathbf{0}
\end{equation*}
                
\item[Description:] All values must be greater or equal to zero.
  Proving the violating condition indicates that this property is broken, and thus the block must not be allowed in the ledger.
        
\end{description}

% !!! TODO: violating conditions

\subsection{Minimum UTxO value}
\label{rule:minimum-utxo-value}
% Same structure as in Cardano L1

\subsection{Network id of inputs}
\label{rule:network-id-of-inputs}

\subsection{Network id of reference inputs}
\label{rule:network-id-of-reference-inputs}

% Reference inputs are needed

\subsection{Network id of outputs}
\label{rule:network-id-of-outputs}

\subsection{Network id of withdrawals}
\label{rule:network-id-of-withdrawals}

\subsection{Network id of transaction}
\label{rule:network-id-of-transaction}

\subsection{All reference inputs must be valid}
\label{rule:all-reference-inputs-must-be-valid}

% TODO: maybe also include $inputs \cap reference\_inputs = \varnothing$

\subsection{Validator scripts are available}
\label{rule:validator-scripts-are-available}

!!! TODO: This is a placeholder, it will only be relevant once Midgard implements phase 2 validations (smart contracts).

\subsection{Validator scripts adhere to execution limits}
\label{rule:validator-scripts-adhere-to-execution-limits}

!!! TODO: This is a placeholder, it will only be relevant once Midgard implements phase 2 validations (smart contracts).

\subsection{Validator scripts accept transaction}
\label{rule:validator-scripts-accept-transaction}

!!! TODO: This is a placeholder, it will only be relevant once Midgard implements phase 2 validations (smart contracts).
% Maybe separate out staking and minting scripts in a separate rule

\subsection{Maximum number of collateral inputs}
\label{rule:maximum-number-of-collateral-inputs}

!!! TODO: This is a placeholder, it will only be relevant once Midgard implements phase 2 validations (smart contracts).

\end{document}
