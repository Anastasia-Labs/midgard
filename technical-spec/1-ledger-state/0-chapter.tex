\documentclass[../midgard.tex]{subfiles}
\graphicspath{{\subfix{../images/}}}
\begin{document}

\chapter{Ledger state}
\label{h:ledger-state}

Midgard's L2 ledger consists of a chain of blocks.
Each block defines a transition from the previous block's set of unspent transaction outputs (utxos) to a new utxo set.
Unlike Cardano's closed-system L1 ledger, Midgard's open-system L2 ledger allows block transitions to create and destroy utxos in response to exogenous events---namely, deposits and withdrawal requests that occur on the L1 ledger.
This is in addition to Midgard's endogenous L2 transactions, which work the same as L1 transactions but with reduced functionality (no staking/governance actions).\footnote{Cardano developers concerned about the absence of the "Withdraw~0" action need not worry.
Midgard's ledger rules permit the "Observe" script purpose defined in \href{https://github.com/cardano-foundation/CIPs/tree/master/CIP-0112}{CIP-112}, which is a more principled replacement for "Withdraw~0" that is expected to arrive in the next era of Cardano mainnet in 2025.}

The blocks are stored temporarily on Midgard's data availability layer and permanently on Midgard's archival nodes.
The blocks' headers are committed to Midgard's L1 state queue data structure to establish immutability for the blocks' sequence and contents as part of Midgard's L1 contract-based consensus protocol.
Block headers have a fixed byte size regardless of how many deposit, transaction, and withdrawal events are held by their blocks---this size leverage between blocks and headers is how Midgard multiplies Cardano's transaction throughput.

Midgard L1 contract-based consensus protocol irreversibly considers a block to be confirmed as soon as all its predecessors are confirmed and one of the following holds:
\begin{description}
    \item[Optimistic confirmation.] The block's maturity period has elapsed without any fraud proof being verified on L1 to prove that the block violates one of Midgard's ledger rules.
    \item[Non-optimistic confirmation.] A compliance proof has been verified on L1 to prove that the block complies with all Midgard ledger rules.
\end{description}

The irreversibility of block confirmation allows the L1 representation of confirmed block headers to be condensed.
For instance, it can stop tracking confirmed withdrawal requests after they are paid out and confirmed deposits after they are absorbed into Midgard's reserve.
It can avoid tracking any confirmed transactions and only needs to track the last confirmed block's utxo set, as it is required to confirm the next block.
Finally, it can drop all other header data for the last confirmed block's predecessors, as it is implicitly tracked by a chained hash in the last confirmed header and is not required to confirm the next block.

As a result, Midgard's L1 confirmed state consists of a fixed-byte-size record with selected fields from the last confirmed block header and a variable-size dataset tracking confirmed withdrawal requests and deposits until they are processed.

\end{document}
