\documentclass[../midgard.tex]{subfiles}
\graphicspath{{\subfix{../images/}}}
\begin{document}

\section{Block}
\label{h:block}

A block consists of a header and a block body:
\begin{equation*}
    \T{Block} \coloneq \left\{
    \begin{array}{ll}
        \T{header} : & \T{Header} \\
        \T{block\_body} : & \T{BlockBody}
    \end{array} \right\}
\end{equation*}

A block body contains Merkle Patricia Tries (MPTs) for the block's transactions, deposits, and withdrawals, along with an MPT for the utxo set that results from applying the block's transition to the previous block's utxo set.
\begin{equation*}
    \T{BlockBody} \coloneq \left\{
    \begin{array}{ll}
        \T{utxo} : & \T{MPT(UtxoSet)} \\
        \T{tx} : & \T{MPT(Tx)} \\
        \T{deposit} : & \T{MPT(DepositEvent)} \\
        \T{withdrawal} : & \T{MPT(WithdrawalEvent)} \\
    \end{array} \right\}
\end{equation*}

\Cref{fig:block-transition} shows how the block's utxo set is derived by applying the block's withdrawals, transactions, and then deposits to the previous block's utxo set. \Cref{fig:block-tx-mpt} shows an example MPT for block transactions.

A block header is a record with fixed-size fields: integers, hashes, and fixed-size bytestrings.
\begin{equation*}
    \T{Header} \coloneq \left\{
    \begin{array}{ll}
        \T{prev\_utxo\_root} : & \T{MPTR} \\
        \T{utxo\_root} : & \T{MPTR} \\
        \T{tx\_root} : & \T{MPTR} \\
        \T{deposit\_root} : & \T{MPTR} \\
        \T{withdraw\_root} : & \T{MPTR} \\
        \T{start\_time} : & \T{PosixTime} \\
        \T{event\_start\_time} : & \T{PosixTime} \\
        \T{end\_time} : & \T{PosixTime} \\
        \T{prev} : & \T{Option(HeaderHash)} \\
        \T{operator\_vkey} : & \T{VerificationKey} \\
        \T{protocol\_version} : & \T{Int} \\
        \T{utxo\_count} : & \T{Int} \\
        \T{tx\_count} : & \T{Int} \\
        \T{deposit\_count} : & \T{Int} \\
        \T{withdraw\_count} : & \T{Int}
    \end{array} \right\}
\end{equation*}

These header fields are interpreted as follows:
\begin{itemize}
    \item The \code{*\_root} and \code{*\_count} fields are the root hashes (MPTRs) and element counts of the corresponding MPTs in the block body.
    \item The \code{prev\_utxo\_root} is a copy of the \code{utxo\_root} from the previous block, included for convenience in the fraud proof verification procedures.
    \item The \code{start\_time} and \code{end\_time} fields are the bounds of the block commitment transaction's time-validity interval.
    \item The \code{event\_start\_time} and \code{end\_time} fields are the bounds of the block's event interval.
    \item The \code{prev} field is a hash of the previous block header. It is empty for the ledger's genesis block but must be set for all subsequent blocks.
    \item The \code{operator\_vkey} field is the cryptographic verification key for signatures of the operator who committed the block header to the L1 state queue.
    \item The \code{protocol\_version} is the Midgard protocol version that applies to this block.
\end{itemize}

\begin{figure}[htb] % place the figure ’here’ or at the page top
    \centering % center the figure
    \begin{tikzpicture}[node distance=2cm]
        \node (s1) [initialstate] {\codeNC{prev\_utxo}};
        \node (s2) [finalstate, right of = s1, xshift=9cm]
            {\codeNC{utxo}};
        
        \path[every node/.style={font=\sffamily\normalsize}] [arrow]
            (s1.south east) edge [bend right = 30]
                 node [anchor = north, above, yshift = 0.5em] {\codeNC{withdrawal}} ($(s1.south east)!0.33!(s2.south west)$)
            ($(s1.south east)!0.33!(s2.south west)$) edge [bend right = 30]
                 node [anchor = north, above, yshift = 0.5em] {\codeNC{tx}} ($(s1.south east)!0.66!(s2.south west)$)
            ($(s1.south east)!0.66!(s2.south west)$) edge [bend right = 30]
                 node [anchor = north, above, yshift = 0.5em] {\codeNC{deposit}} (s2.south west)
            (s1.north east) edge [bend left = 15]
                node [anchor = south, below, yshift = -0.5em] {Block transition} (s2.north west);
    \end{tikzpicture}
    \caption{A block's transition from a previous block's utxo set to a new utxo set.}
    \label{fig:block-transition}
\end{figure}

\begin{figure}[htb]
\centering
\begin{tikzpicture}[
    level distance=2.5cm,
    sibling distance=5cm,
    edge from parent/.style={draw, -{Circle[open]}, thick},
    every node/.style={draw, thick, rounded corners, align=center, fill=orange!20, font=\sffamily\small},
    branch/.style={fill=gray!20},
    leaf/.style={fill=green!20},
    level 1/.style={sibling distance=7cm},  % Increased distance for first level
    level 2/.style={sibling distance=4cm}   % Adjusted distance for second level
]

% Root node
\node {tx\_root}
    % First level branches
    child { node[branch] {Branch 1\\ 00}
        % Second level branches
        child { node[leaf] {Leaf\\ 000 \\ Transaction 1} }
        child { node[branch] {Branch 2\\ 01}
            % Third level branches
            child { node[leaf] {Leaf\\ 010 \\ Transaction 2} }
            child { node[leaf] {Leaf\\ 011 \\ Transaction 3} }
        }
    }
    child { node[branch] {Branch 3\\ 11}
        % Second level branches
        child { node[leaf] {Leaf\\ 110 \\ Transaction 4} }
        child { node[leaf] {Leaf\\ 111 \\ Transaction 5} }
    };

\end{tikzpicture}
\caption{A Merkle Patricia Trie example for a block's transactions. The block header's \code{tx\_root} is shown at the top in orange.}
\label{fig:block-tx-mpt}
\end{figure}

\end{document}
